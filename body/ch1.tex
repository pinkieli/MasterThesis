\chapter{绪论}
%
\section{课题来源}
根据导师在研一期间的指导及个人兴趣、基础知识的储备出发,通过查阅相关资料并在导师的指导下共同商定此论文题目。

在大四毕业后的暑假期间根据导师推荐的几篇文章\cite{杨超2015},我开始了学习结构动力学运动方程数值求解算法。同时,在研一上学期由刘伟老师讲解的《高等结构动力学》课程中,我再一次接触到了结构动力学方程的数值求解。同时在该课程实验中,我也第一次发现了满足一些看似可行的直接积分法,求解出的响应有时却并不可靠,例如Wilson-$\theta$法\sindex[china]{Wilson-$\theta$!Wilson-$\theta$}。它的这种特性由于其本身的数值耗散和弥散特性造成的。这次学习让我对求解结构动力学方程的数值算法产生了兴趣。同时,我也知道力学对数学,尤其是计算数学的要求是很高的。于是在研究生期间,又选了很多数学专业的核心课程(非线性数值分析、现代常微分方程理论、小波分析)进行学习,为自己后面进一步的探究数值算法铺平道路。这些数学课程的学习中,我愈发的觉得结构动力学方程的直接积分法的研究还有很多工作可做。特别值得一提就是,似乎在力学中我们更多的是借鉴了线性多步法来发展直接积分法,而使用Runge-Kutta法的思想发展结构动力学求解算法少之甚少。特别就是由Butcher J. C.在1966年提出的更具一般性的算法框架去求解常微分方程\cite{Butcher1966}。在该框架下,线性多步法和Runge-Kutta法都是其特例。这类算法应用到求解结构动力学运动方程还鲜有文献。这也是进行本文研究的一个主要目标。
\section{课题研究背景、目的和意义}
\subsection{结构动响应问题的工程背景}
动力学问题在国民经济和科学技术的发展中有着广泛的应用领域。最常见遇到的是结构动力学问题,它有两类研究对象。一类是在动力状态下工作的机械或结构,例如高速旋转的电机,汽轮机,离心压缩机,冲压机床,以及高速运行的车辆,飞行器等,它们承受本身惯性及与周围介质或结构相互作用的动力载荷。如何保证它们运行的平稳性及结构的安全性,是极为重要的课题。另一类是承受动力载荷作用的工程结构,例如建于地面的高层建筑,反应塔和管道,核电站的安全壳和热交换器。近海工程的海洋石油平台等,他们可能承受强风,水流,地震以及波浪等各种动力载荷的作用。这些结构的破裂、倾覆和垮塌等事故的发生,将给人民的生命财产造成巨大的损失。正确分析和设计这类结构,在理论上和实际上都是具有重要意义的课题。动力学研究的另一重要领域是波在介质中的传播问题。它是研究短暂作用于介质边界或内部的载荷所引起的位移和速度的变化,如何在介质中向周围传播,以及在界面上如何反射,折射等规律。它的研究在结构的抗震设计,人工地震勘测,无损探伤等领域都有广泛的应用背景,因此也是近二十都年一直受到工程和科技界密切关注的课题。
\subsection{结构动响应数值算法的数学力学背景}
一般情况下,大多数工程问题的数学建模后得到的微分方程往往是不可能求出解析解的;或者说,花费很大的努力求得的解析解是不经济的。这也导致了在实际应用中,应用数值算法求解得到的微分方程是一种必要手段。在结构动力学运动方程的求解过程中更是这种情况。

结构动力学数值计算主要有两大类传统方法\cite{于开平2005}。一是在空间域使用有限元离散后,基于大多数工程结构动态响应主要以低频为主的假设,使用模态分解和叠加的步骤,给出模态截断后的动响应,可归类于近似解析法。这类方法适合于比例阻尼假设情况,适合于长时间、持续动载荷作用问题,以及大多数以低频响应为主的工程问题。但对中高频激励问题(基于有限元模型,航天工程更关注的中高频多数指200-2000HZ范围)的计算尚没有很好解决。主要原因是:2000HZ以内可能有很多阶模态,高阶模态对参数变化更为敏感,同时要求更细密的有限元网格\cite{于开平2005}。此外,高阶模态阻尼比的确定也还没有合理的模型,存在精度和计算量两方面的问题。另外一种方法是在时间域使用有限差分,或者使用时间有限元离散,得到时间逐步递推的计算步骤,还有一类是时空域同时离散的时空有限元方法。这类方法属于数值方法,不仅适用于比例阻尼也适用于于非比例阻尼、非线性情况。这类方法也将是本文分析和研究的重点。

目前在大型结构的瞬态动力学、非线性动力学响应数值计算问题上,比较有效的方法仍然是时域内的直接积分方法,又称逐步积分法。该类方法,有合适的精度、合适的计算量以及适合于大多数实际工程问题。也就是说算法数值计算的整体性能好、适用范围广,因此一直受到计算力学、计算数学工作者和工程界的重视。

\subsection{课题研究的意义}
目前,用线性多步法的理论去发展直接积分法的理论比较完善。特别是经过Dahlquist的发展推广\cite{Dahlquist1956},得到了如下结论\cite{book:dover}:
\begin{itemize}
\item[(1)] 	一个显式的、A-稳定的线性多步法是不存在的。
\item[(2)] 	一个三阶及其以上精度的A-稳定的线性多步法是不存在的。
\item[(3)] 带有最小误差常数的二阶精度的A-稳定的线性多步法是梯形积分规则。
\end{itemize}
于是,想要利用线性多步法的理论去发展具有A-稳定的算法已经是不可能的。然而,在数学分支—常微分方程数值解的理论中,我们也得知通过适当的构造,广义线性法可以突破上述线性多步法的障碍,理论上可以实现任意阶精度下,其数值方法仍能保持一定好的数值特性,例如A-稳定性,甚至L-稳定性。特别地,作为广义线性法的一个特例—Runge-Kutta法,也可以实现上述特性。故本文的研究意义在于,将广义线性法应用到结构动力学运动方程求解中,进而推导出新的求解格式。这些新的算法格式在数学分析框架下具有良好的性质,但应用到力学问题上,可能还需要进行一定的改进和完善。同时,这些新的数值格式也将在力学背景下进行算法的优劣分析,比如,其数值高频耗散、相对周期误差等。特别地,本文也将尝试利用广义线性法中对于非线性常微分方程、刚度问题的分析借鉴到力学中的非线性动力学、刚度硬化问题的分析上。这样可以对现有直接积分法在理论分析非线性问题时,提供一个理论判断依据。
\section{国内外研究现状}
目前,国内外在广义线性法和结构动力学数值算法的发展中,几乎都是相对独立的。鲜有研究者将者二者结合起来。但过去的几十年也有一些研究学者将Runge-Kutta法应用到结构动力学数值求解中,进而构造了一些新的数值算法。

周树荃和高科华\cite{周树荃1992}在论文中利用3级3阶半隐式Runge-Kutta法求解结构动力学问题,并利用多项式预处理共轭梯度法求解有关代数方程组,提出了半隐式RK型并行直接积分法(RK33P).与相应的串行算法RK33S算法进行比较,发现当求解系统的阶数为103-104,其加速度比可达24-27。

Christoph L.和Simeon B.\cite{Lunk2004,Lunk2005}利用Runge-Kutta-Nyström方法求解动力学中的刚性力学系统,该方法一般化了著名的Störmer规则,并且使得求解的稳定域最大化。结果表明,文章提出的RKN方法具有一定的可比性和高效性。

Dopico D.和Lugris U.\cite{Dopico2010}等分析了在实时多体结构系统中具有遗传Runge-Kutta稳定性的广义线性方法(IRK)的两个应用。文章问答了IRK方法是否适合求解实时结构响应以及该类方法是否比传统的Newmark家族方法更优。

吴志桥等人\cite{吴志桥2010}将几种具有不同稳定性的Runge-Kutta方法应用到结构动力学的数值求解中,并且使用了减小计算量的两种方法:使用单对角隐式Runge-Kutta方法和应用转化矩阵,算例表明在精确解上较小的物理阻尼能有效抑制高频分量,但对各种直接积分方法的影响较小,较高精度的L-稳定Runge-Kutta方法能有效抑制高频分量的同时高精度的求解低频振动。而且在他的博士论文\cite{吴志桥2009}中,通过研究Runge-Kutta方法对结构动力学运动方程的求解格式,得到A-稳定与直接积分法中的无条件稳定是等价的,而L-稳定格式包含数值阻尼。

Yin S. H.\cite{Yin2013}通过在结构动力学数值求解格式中构造位移和速度显式更新公式,推导出带四个参数的放大矩阵,同时利用Runge-Kutta方法求解对应的结构动力学运动微分方程得到其状态向量的迭代格式,进而得到其迭代矩阵,最后通过让放大矩阵和迭代矩阵对应风量相等,得到其位移速度更新公式中的四个参数,结果表明其参数是结构依赖的。

郭静和邢誉峰\cite{郭静2014}在2014年利用2级4阶隐式Guass-Lengendre辛Runge-Kutta方法(GLSRK)求解有阻尼和外荷载情况下的线性动力学运动方程。并首次给出了Guass-Lengendre辛RK方法和经典RK方法的谱半径和单步相位误差的显式表达式。通过算例表明,辛RK方法比经典RK方法优越,尤其是在运动学特性和长时间数值模拟方面尤为显著;但是该算法与传统的Newmark-$\beta$法相比,计算量和储存量过于庞大,这也在一定程度上限制了该方法的应用。

针对上述问题,黄策、富明慧\cite{黄策2016}等人将郭静等人提出的四阶隐式Guass-Lengendre辛RK法进行优化,先通过消元,降低问题的规模,再利用消元后得到的方程系数矩阵正定对称的特点,采用预处理共轭梯度法求解。与原算法相比,改进后的算法大幅度降低了储存量和计算量;同时与Newmark-$\beta$法相比,在未明显增加计算量的前提下,大幅度提高了计算精度。

可以看出,国内外利用Runge-Kutta方法去求解结构动力学方程的研究也是较为零散的,还有大量的工作可做。
\section{本文主要研究内容}
本文将从数学常微分方程数值方法—广义线性法和力学中结构动力学运动方程的结构依赖型直接积分法两个角度出发,针对动力学运动方程的数值解法进行探究。本论文的主要研究内容有:
\begin{itemize}
\item 系统学习和探究广义线性法的体系框架。总结广义线性法至今为止的研究成果和应用领域。同时,以广义线性法的知识框架去进一步学习和研究线性多步法。
\item 对现阶段新兴的的结构依赖型算法进一步系统研究。挖掘新的具有实用价值的算法。同时对利用其它构造思想去发展结构动力学数值解法的文献进行总结和归纳。
\item 将广义线性法的几类主要分支算法进行研究,并尝试将其应用到结构动力学数值求解中,以期望获得很好的数值算法。目前根据现有文献研究表明,这种思路是可行的,但可能需要做进一步的细致工作研究。
\item 将广义线性法对非线性系统的刻画应用到直接积分法求解结构动力学非线性系统的分析。
\item 将上述发展算法进行C++或MATLAB代码实现,得到其相应的工具包。
\end{itemize}