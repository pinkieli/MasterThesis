\chapter{积分算法设计指标}


\section{相容性}

\section{稳定性}

\section{收敛性}

\section{精度分析}

\subsection{数值耗散}

\subsection{数值弥散}

\section{超调行为}

\section{虚假根分析}

\section{直接积分法的放大矩阵}

\begin{definition}
直接积分算法的一致性\cite{Hoff1988}。如果两个直接积分算法的放大矩阵$A$的对应元素分别相等\footnote{这也是矩阵相等的定义。},则称这两个积分算法是一致的。
\end{definition}

\begin{definition}
直接积分算法的相似性/谱等价\cite{Hoff1988}。如果两个直接积分算法的放大矩阵$A$的不变量分别对应相等,则称这两个积分算法是相似的/谱等价的。
\end{definition}

需要说明的是,一般情况下,两个谱等价的算法并不一定能导出一致的数值算法特性;而两个算法的一致性可以保证。这样的例子可以参见文献\inlinecite{Hoff198887}。

\section{本章小结}



