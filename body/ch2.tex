\chapter{积分算法的设计及基本分析}
对于结构动力学运动方程
\begin{equation}
\bm{M}\ddot{\bm{U}}(t)+\bm{C}\dot{\bm{U}}(t)+\bm{KU}(t)=\bm{F}(t)\label{eq:ch2DyEq}
\end{equation}
的数值求解及其数值算法的设计分析,许多的基本概念借助于数学上对一阶微分方程
\begin{equation}
\dot{y}(t)=f(y,t)\label{eq:ch2Firsty}
\end{equation}
的数值算法的设计分析,如数值算法的相容性、稳定性和收敛性等。事实上,结构动力学运动方程(\ref{eq:ch2DyEq})本质上就是一个二阶的线性微分方程,通过引入速度为独立的变量,可以将其二阶微分形式降维为具有形如(\ref{eq:ch2Firsty})形式的更高阶一阶微分方程。因此,在这一节主要针对(\ref{eq:ch2Firsty})进行数值算法性能指标分析,如果一些定义对于二阶微分方程有差异,再具体指出。

由于广义线性法是求解微分方程(\ref{eq:ch2Firsty})更具一般性的方法,因此本节先通过最简单的数值算法(Euler法)分析引出广义线性法,而后后续的算法性能分析都尽可能在广义线性的基础上进行定义及分析。而常见的线性多步法和Runge-Kutta法将是由其特例导出。

\section{广义线性法}
在20世纪初.Euler在他的著作中,针对如下自治的一阶微分方程
\begin{equation}
\dot{y}(t)=f(y(x))\label{eq:ch2FirstAuto}
\end{equation}
带有合适的初始条件$y(x_0)=y_0\in\mathbb{X}$的数值求解,给出了如下的数值格式
\begin{equation}
y_n=y_{n-1}+h_nf(y_{n-1}),\qquad n=1,2,\cdots\label{eq:ch2Euler}
\end{equation}
对于每一个给定的$n$值,其$y_n$都可以根据上式由$y_{n-1}$计算得到。同时,$h_n$为在第$n$的积分步长,一般情况下没必要要求在整个积分过程中相等,亦即变时间步长积分。但在本文整个分析过程中,我们假定其数值算法都是在相等的时间步长内进行积分及计算的。亦即
\begin{equation}
h_n\equiv h\qquad n=1,2,\cdots
\end{equation}

显然,该数值格式的优劣很大程度依赖于积分步长$h$以及等式(\ref{eq:ch2FirstAuto})右端项$f(y)$随变量$y$的变化快慢程度。而后者通常由$f(y)$满足的Lipschitz常数$L$来度量。同时,我们也需要提及该条件也保证了微分方程(\ref{eq:ch2FirstAuto})解的存在唯一性。于是,我们在本全文中都假定分析的微分方程都满足该Lipschitz条件。关于该条件的对其微分方程解的存在唯一性的证明,读者可以翻阅任意一本常微分方程的书籍都可以查到。这里不再做过多累述。

事实上,Euler格式(\ref{eq:ch2Euler})简单易用,但它却是条件稳定的且只有一阶精度。在很多时候,并不是很实用。因此,后续许多学者提出了很多改进Euler格式的算法。其中,至少有两个策略进行改进Euler方法
\begin{itemize}
\item[\ddag] 单步内使用更多、更复杂的计算,如Runge-Kutta法。
\item[\ddag] 使用更多已知节点的逼近值。如线性多步法。
\item[\ddag] 更高阶导数值的使用。如Rosenbrock方法。
\item[\ddag] 多步-多级-多导数方法。如拟Runge-Kutta方法。
\end{itemize}

从简单的Euler格式出发,我们可以建立如图\ref{Fig:ch2GeneraAxis}的改进策略坐标系
\begin{figure}[htpb]
\centering
\begin{tikzpicture}[xscale=1.2,yscale=1.2]
\draw [thin,->] (0,0) -- (2,1) node [align=center,below right] {每步更多的\\ 计算量};
\draw [thin,->] (0,0)--(-2,1.5) node [align=center,below left] {更多的\\ 已知节点值};
\draw [thin,->] (0,0)--(0,2) node [left] {$y$的导数};
\draw [thin,->] (0,2) -- (0,4) node [left] {$f$的导数};
%============================================
\draw [thin,fill,red] (0,0) circle [radius = 0.03] node [below] {Euler格式};
\end{tikzpicture}
\bicaption[Fig:ch2GeneraAxis]{}{Euler格式的改进策略示意图}{Fig.$\!$}{Generalizations of the Euler method}
\end{figure}

在图\ref{Fig:ch2GeneraAxis}的坐标系下,现阶段的大多数的数值算法都可以被建立在这个三维算法格式空间中,如图\ref{Fig:ch2GeneraEuler}所示。
\begin{figure}[htpb]
\centering
\begin{tikzpicture}[xscale=1.2,yscale=1.2]
\draw [thin] (0,0) -- (2.5,1) node [right] {Runge-Kutta法};
\draw [thin] (0,0)--(-2,1.5) node [left] {线性多步法};
\draw [thin,dashed] (-2,1.5)--(0.5,2.5);
\draw [thin,dashed] (2.5,1) -- (0.5,2.5);
%============================================
\draw [thin] (0,3) node [left] {泰勒级数法} -- (2.5,4);
\draw [thin] (0,3)--(-2,4.5) node [left] {Obreshkov方法};
\draw [thin,dashed] (-2,4.5)--(0.5,5.5);
\draw [thin,dashed] (2.5,4) -- (0.5,5.5);
%--------------------------------
\draw [thin] (0,3) -- (0,0);
\draw [thin] (2.5,4) -- (2.5,1);
\draw [thin] (-2,4.5) -- (-2,1.5);
\draw [thin,dashed] (0.5,2.5) -- (0.5,5.5);
%=-==========================================
\draw [thin] (0,6) -- (2.5,7) node [right] {Rosenbrock方法};
\draw [thin] (0,6)--(-2,7.5);
\draw [thin] (-2,7.5)--(0.5,8.5);
\draw [thin] (2.5,7) -- (0.5,8.5);
%--------------------------------
\draw [thin] (0,6) -- (0,3);
\draw [thin] (2.5,7) -- (2.5,4);
\draw [thin] (-2,7.5) -- (-2,4.5);
\draw [thin,dashed] (0.5,5.5) -- (0.5,8.5);
%=-==========================================
\draw [thin,fill,red] (0,0) circle [radius = 0.03] node [below] {Euler格式};
\draw [thin,fill,blue] (0.5,2.5) circle [radius = 0.03] node [right] {广义线性法};
\draw [thin,fill] (0,3) circle [radius = 0.03];
\draw [thin,fill] (2.5,7) circle [radius = 0.03];
\draw [thin,fill] (-2,4.5) circle [radius = 0.03];
\draw [thin,fill] (2.5,1) circle [radius = 0.03];
\draw [thin,fill] (-2,1.5) circle [radius = 0.03];
\end{tikzpicture}
\bicaption[Fig:ch2GeneraEuler]{}{Euler格式的改进算法框架}{Fig.$\!$}{Generalized formworks of the Euler method}
\end{figure}
于是,可以知道广义线性法是线性多步法和Runge-Kutta法的结合,更是Euler格式的一般化推广。同时,从该图可以知道,广义线性法并不能包括所有的数值格式,它仅仅是现阶段数值算法的一大类。

广义线性方法即结合了线性多步法的多值特点,又使用了Runge-Kutta方法的多级属性。并且它首次由Butcher教授在1966年提出\cite{Butcher1966}。下面提及的广义线性法的矩阵表示形式则是由Burrage和Butcher在1980年引入而来\cite{Burrage1980}。

假定在单步内进行转换的量有$r$个。在第$n$步的开始,这$r$个量被表示为$y_1^{[n-1]},y_2^{[n-1]},\cdots,y_r^{[n-1]}$。当第$n$步完成计算时,对应的$r$量分别为$y_1^{[n]},y_2^{[n]},\cdots,y_r^{[n]}$,并且这些量将作为下一个时间步内的起始值进行后续计算。同时,在单步内计算的$s$个级数值$Y_1,Y_2,\cdots,Y_s$所对应的级数导数值为$F_1,F_2,\cdots,F_S$。为了表示方便,引入如下的$r$或$s$维向量表示,即
\begin{equation}
y^{[n-1]}=\begin{bmatrix}
y_1^{[n-1]}\\
y_2^{[n-1]}\\
\vdots\\
y_r^{[n-1]}
\end{bmatrix},\quad
y^{[n]}=\begin{bmatrix}
y_1^{[n]}\\
y_2^{[n]}\\
\vdots\\
y_r^{[n]}
\end{bmatrix},\quad
Y^{[n]}=\begin{bmatrix}
Y_1^{[n]}\\
Y_2^{[n]}\\
\vdots\\
Y_s^{[n]}
\end{bmatrix},\quad
F(Y^{[n]})=\begin{bmatrix}
f(Y_1^{[n]})\\
f(Y_2^{[n]})\\
\vdots\\
f(Y_s^{[n]})
\end{bmatrix}
\end{equation}

类似于Runge-Kutta方法,级数值$Y_i$是依赖于级数导数值$F_i$的线性组合来计算,但是现在的广义线性法将其推广为不仅依赖于级数导数值的线性组合,也依赖于前述已知节点逼近值$y_i$的线性组合。即
\begin{equation}
Y_i=\sum_{j=1}^{s}ha_{ij}f(Y_j^{[n]})+\sum_{j=1}^{r}u_{ij}y_j^{[n-1]},\qquad i=1,2,\cdots,s
\end{equation}
同理,对于输出量$y_i^{[n]}$也是不仅线性依赖于各个级数导数值$F_i=f(Y_i)$而且也是线性依赖于前述已知节点逼近值$y_i$,即
\begin{equation}
y_i^{[n]}=\sum_{j=1}^{s}hb_{ij}f(Y_j^{[n]})+\sum_{j=1}^{r}v_{ij}y_j^{[n-1]},\qquad i=1,2,\cdots,r
\end{equation}

令$A=[a_{ij}]_{s\times s},U=[u_{ij}]_{s\times r},B=[b_{ij}]_{r\times s}$以及$V=[v_{ij}]_{r\times r}$,同时使用Kronecker积符号($\otimes$),广义线性法则可以表示为
\begin{align}
Y^{[n]}&=h(A\otimes I)F(Y^{[n]})+(U\otimes I)y^{[n-1]}\\
y^{[n]}&=h(B\otimes I)F(Y^{[n]})+(V\otimes I)y^{[n-1]}
\end{align}
其中,$I$表示维度相容的单位矩阵。而Kronecker积定义如下,若$A\in\mathbb{R}^{m_1\times n_1}$和$B\in\bm{R}^{m_2\times n_2}$,则有
\begin{equation}
A\otimes B=\begin{bmatrix}
a_{11} B&&a_{12}B&&\cdots && a_{1,n_1} B\\
a_{21}B&&a_{22}B&&\cdots && a_{2,n_1}B\\
\vdots &&\vdots && &&\vdots\\
a_{m_1,1}B &&a_{m_1,2}B&&\cdots &&a_{m_1,n_1}B
\end{bmatrix}\in\mathbb{R}^{m_1m_2\times n_1n_2}
\end{equation}
进一步,广义线性法可以被表达为如下形式
\begin{equation}
\begin{bmatrix}
\begin{array}{c}
Y^{[n]}\\ \hline
y^{[n]}
\end{array}
\end{bmatrix}=\begin{bmatrix}
\begin{array}{c|c}
A\otimes I & U\otimes I \\ \hline
B\otimes I & V\otimes I
\end{array}
\end{bmatrix}\begin{bmatrix}
\begin{array}{c}
hF(Y^{[n]})\\ \hline
y^{[n-1]}
\end{array}
\end{bmatrix}\label{eq:ch2GLM}
\end{equation}

通常情况下,广义线性法的数值性能就被这四个矩阵所决定,即$A,U,B$和$V$。因此对于一个广义线性法,可以用如下的分块矩阵刻画
\begin{equation}
\begin{bmatrix}
\begin{BMAT}[5pt]{c:c}{c:c}
	\bm{A} & \bm{U} \\
	\bm{B} & \bm{V}
\end{BMAT}
\end{bmatrix}
\end{equation}

当四个矩阵取值不同时,对应着不同的广义线性法。特别地,线性多步法和Runge-Kutta都是其特例。
\subsection{线性多步法}
考虑对于一阶微分方程(\ref{eq:ch2Firsty})的$k$步线性多步法
\begin{equation}
y_n=\sum_{j=1}^{k}\alpha_jy_{n-j}+h\sum_{j=0}^{k}\beta_jf(y_{n-j})\label{eq:ch2lms}
\end{equation}
令$Y^{[n]}=y_n$,并且
\begin{subequations}
\begin{align}
y^{[n]}&=[y_n,y_{n-1},\cdots,y_{n-k+1},hf(y_n),hf(y_{n-1}),\cdots,hf(y_{n-k+1})]^{\text{T}}\\
y^{[n-1]}&=[y_{n-1},y_{n-2},\cdots,y_{n-k},hf(y_{n-1}),hf(y_{n-2}),\cdots,hf(y_{n-k})]^{\text{T}}
\end{align}
\end{subequations}
于是,对于$k$步的线性多步法(\ref{eq:ch2lms})可以写成广义线性法的形式\cite{Burrage1980},即$r=2k,s=1$
\begin{equation}
\begin{bmatrix}
\begin{BMAT}[5pt]{c:c}{c:c}
A & U\\
B & V
\end{BMAT}
\end{bmatrix}=\begin{bmatrix}
\begin{BMAT}[5pt]{c:cccccccc}{c:cccccccc}
\beta_0 & \alpha_1 & \cdots & \alpha_{k-1} & \alpha_k & \beta_1 & \cdots & \beta_{k-1} & \beta_k\\
\beta_0 & \alpha_1 & \cdots & \alpha_{k-1} & \alpha_k & \beta_1 & \cdots & \beta_{k-1} & \beta_k\\
0 & 1 & \cdots & 0 & 0 & 0 & \cdots & 0 & 0\\
\vdots & \vdots & \ddots &\vdots & \vdots &\vdots &\ddots &\vdots &\vdots \\
0 & 0 &\cdots & 1 & 0 &0 &\cdots & 0 & 0 \\
1 & 0 & \cdots & 0 & 0 & 0 & \cdots & 0 & 0\\
0 & 0 & \cdots & 0 & 0 & 1 & \cdots & 0 & 0\\
\vdots & \vdots & \ddots &\vdots & \vdots &\vdots &\ddots &\vdots &\vdots \\
0 & 0 & \cdots & 0 & 0 & 0 & \cdots & 1 & 0
\end{BMAT}
\end{bmatrix}
\end{equation}
特别地,$k$步的线性多步法(\ref{eq:ch2lms})也可以写成$r=k,s=1$的广义线性法形式\cite{Butcher2006c}。文\inlinecite{Butcher2006c}中定义
\begin{equation}
y_i^{[n-1]}=\sum_{j=k-i+1}^{k}(\alpha_jy_{n+k-i-j}+h\beta_jf(y_{n+k-i-j})),\quad i=1,2,\cdots,k\label{eq:ch2skeel}
\end{equation}
其实,公式(\ref{eq:ch2skeel})在文献\inlinecite{Skeel1979}就已经被提出来了,只不过没有涉及广义线性法的应用。于是,线性多步法(\ref{eq:ch2lms})则可以写成如下形式
\begin{equation}
y_n=h\beta_0f(y_n)+\sum_{j=1}^{k}(\alpha_jy_{n-j}+h\beta_jf(y_{n-j}))=h\beta_0f(y_n)+y_k^{[n-1]}
\end{equation}
进一步,在$n$步结束时有
\begin{equation}
\begin{aligned}
y_i^{[n]}&=\sum_{j=k-i+1}^{k}(\alpha_jy_{n+1+k-i-j}+h\beta_jf(y_{n+1+k-i-j}))\\
&=\alpha_{k-i+1}y_n+h\beta_{k-i+1}f(y_n)+\sum_{j=k-i+2}^{k}(\alpha_jy_{n+1+k-i-j}+h\beta_jf(y_{n+1+k-i-j}))\\
&=(\alpha_{k-i+1}\beta_0+\beta_{k-i+1})hf(y_n)+\alpha_{k-i+1}y_k^{[n-1]}+y_{i-1}^{[n-1]}
\end{aligned}
\end{equation}
其中,$i=1,2,\cdots,k$。线性多步法(\ref{eq:ch2lms})的另外一种广义线性法的表出形式为
\begin{equation}
\begin{bmatrix}
\begin{BMAT}[5pt]{c:c}{c:c}
A & U\\
B & V
\end{BMAT}
\end{bmatrix}=\begin{bmatrix}
\begin{BMAT}[5pt]{c:cccccc}{c:cccccc}
\beta_0 & 0 & 0 & 0 & \cdots & 0 & 1\\
\alpha_k\beta_0+\beta_k & 0 & 0 & 0 & \cdots & 0 & \alpha_k\\
\alpha_{k-1}\beta_0+\beta_{k-1} & 1 & 0 & 0 & \cdots & 0 & \alpha_{k-1}\\
\alpha_{k-2}\beta_0+\beta_{k-2} & 0 & 1 & 0 & \cdots & 0 & \alpha_{k-2}\\
\vdots & \vdots & \vdots & \vdots & \ddots & \vdots & \vdots \\
\alpha_{2}\beta_0+\beta_{2} & 0 & 0 & 0 & \cdots & 0 & \alpha_{2}\\
\alpha_{1}\beta_0+\beta_{1} & 0 & 0 & 0 & \cdots & 1 & \alpha_{1}\\
\end{BMAT}
\end{bmatrix}
\end{equation}
其中,$Y^{[n]}=y_n$,而$y^{[n]}$则为
\begin{equation}
y^{[n]}=[y_1^{[n]},y_2^{[n]},\cdots,y_{k-1}^{[n]},y_{k}^{[n]}]^{\text{T}}
\end{equation}
\subsubsection{Adams算法}
线性多步法中,求解非刚性问题时最常用的方法就是Adams方法族。对于线性多步法公式(\ref{eq:ch2lms})中,令
\begin{equation}
\alpha_1=1,\qquad \alpha_i=0,\ i>1
\end{equation}
于是,Adams算法的一般形式为
\begin{equation}
y_n=y_{n-1}+h\sum_{j=0}^{k}\beta_jf(y_{n-j})\label{eq:ch2Adams}
\end{equation}
需要说明的是,算法(\ref{eq:ch2Adams})的显式形式通常称为Adams-Bashforth算法;而其隐式算法则称为Adams-Moulton算法。

于是将Adams算法改写成广义线性法的形式,则有
\begin{equation}
\begin{bmatrix}
\begin{BMAT}[4.5pt]{c}{c:cccccc}
Y_1\\
y_n\\
hf(y_n)\\
hf(y_{n-1})\\
hf(y_{n-2})\\
\vdots\\
hf(y_{n-k+1})
\end{BMAT}
\end{bmatrix}=\begin{bmatrix}
\begin{BMAT}[5pt]{c:cccccc}{c:cccccc}
\beta_0 & 1 & \beta_1 & \beta_2 & \cdots & \beta_{k-1} & \beta_k\\
\beta_0 & 1 & \beta_1 & \beta_2 & \cdots & \beta_{k-1} & \beta_k\\ 
1		& 0 & 0		  & 0   &\cdots &0 &0\\
0		& 0 & 1		  & 0   &\cdots &0 &0\\
0		& 0 & 0		  & 1   &\cdots &0 &0\\
\vdots & \vdots & \vdots & \vdots & \ddots & \vdots & \vdots\\
0	& 0 & 0		  & 0   &\cdots &1 &0
\end{BMAT}
\end{bmatrix}\begin{bmatrix}
\begin{BMAT}[4.5pt]{c}{c:cccccc}
hf(Y_1)\\
y_{n-1}\\
hf(y_{n-1})\\
hf(y_{n-2})\\
hf(y_{n-3})\\
\vdots\\
hf(y_{n-k})
\end{BMAT}
\end{bmatrix}\label{eq:ch2GLMAdams}
\end{equation}

向前显式Euler法对应的参数为
\begin{equation}
k=1,\quad \beta_0=0,\quad\beta_1=1
\end{equation}
于是,从等式(\ref{eq:ch2GLMAdams})可知其在广义线性法框架下的表示形式为
\begin{equation}
\begin{bmatrix}
\begin{BMAT}[5pt]{c:cc}{c:ccc}
0 & 1 &1\\
0 & 1 & 1\\
1 & 0 & 0\\
0 & 0 & 1
\end{BMAT}
\end{bmatrix}
\end{equation}
事实上,向前显式Euler法也可以写为如下更加简洁的形式
\begin{equation}
\begin{bmatrix}
\begin{BMAT}[5pt]{c}{c:c}
Y_1\\
y_n
\end{BMAT}
\end{bmatrix}=\begin{bmatrix}
\begin{BMAT}[5pt]{c:c}{c:c}
0 & 1\\ 1 & 1
\end{BMAT}
\end{bmatrix}\begin{bmatrix}
\begin{BMAT}[5pt]{c}{c:c}
hf(Y_1)\\
y_n
\end{BMAT}
\end{bmatrix}
\end{equation}
其中,$Y_1=y_{n-1}$。

尽管Adams-Moulton算法是隐式的,由于其较小的稳定域,它们也仅仅只在求解非刚性问题时使用。除此之外,它们也常常在作为预测-校正格式使用。
\subsubsection{预测-校正格式}
预测-校正格式中,使用Adams-Bashforth方法作为一个预测逼近算法,而Adams-Moulton作为校正格式算法。同时,预测-校正格式通常简写为“PEC”或者“PECE”,其中“P”代表预测格式,而“E”表示计算,“C”表示校正格式。其算法一般形式可写为
\begin{subequations}
\begin{align}
y_n^*&=y_{n-1}+h\sum_{j=1}^{k}\beta_j^*f(y_{n-j})\\
y_n&=y_{n-1}+h\beta_0f(y_n^*)+h\sum_{j=1}^{k}\beta_jf(y_{n-j})
\end{align}
\end{subequations}

于是,一个PEC格式可以由下列的广义线性法表示
\begin{equation}
\begin{bmatrix}
\begin{BMAT}[4.5pt]{c}{c:cccccc}
Y_1\\
y_n\\
hf(y_n)\\
hf(y_{n-1})\\
hf(y_{n-2})\\
\vdots\\
hf(y_{n-k+1})
\end{BMAT}
\end{bmatrix}=\begin{bmatrix}
\begin{BMAT}[5pt]{c:cccccc}{c:cccccc}
0 & 1 & \beta_1^* & \beta_2^* & \cdots & \beta_{k-1}^* & \beta_k^*\\
\beta_0 & 1 & \beta_1 & \beta_2 & \cdots & \beta_{k-1} & \beta_k\\ 
1		& 0 & 0		  & 0   &\cdots &0 &0\\
0		& 0 & 1		  & 0   &\cdots &0 &0\\
0		& 0 & 0		  & 1   &\cdots &0 &0\\
\vdots & \vdots & \vdots & \vdots & \ddots & \vdots & \vdots\\
0	& 0 & 0		  & 0   &\cdots &1 &0
\end{BMAT}
\end{bmatrix}\begin{bmatrix}
\begin{BMAT}[4.5pt]{c}{c:cccccc}
hf(Y_1)\\
y_{n-1}\\
hf(y_{n-1})\\
hf(y_{n-2})\\
hf(y_{n-3})\\
\vdots\\
hf(y_{n-k})
\end{BMAT}
\end{bmatrix}
\end{equation}
其中,$Y_1=y_n^*$。而PECE格式则可以写为
\begin{subequations}
\begin{align}
y_n^*&=y_{n-1}+h\sum_{j=1}^{k}\beta_j^*f(y_{n-j})\\
y_n&=y_{n-1}+h\beta_0f(y_n^*)+h\sum_{j=1}^{k}\beta_jf(y_{n-j})
\end{align}
\end{subequations}

于是,一个PEC格式可以由下列的广义线性法表示
\begin{equation}
\begin{bmatrix}
\begin{BMAT}[4.5pt]{c}{cc:cccccc}
Y_1\\
Y_2\\
y_n\\
hf(y_n)\\
hf(y_{n-1})\\
hf(y_{n-2})\\
\vdots\\
hf(y_{n-k+1})
\end{BMAT}
\end{bmatrix}=\begin{bmatrix}
\begin{BMAT}[5pt]{cc:cccccc}{cc:cccccc}
0 & 0 & 1 & \beta_1^* & \beta_2^* & \cdots & \beta_{k-1}^* & \beta_k^*\\
\beta_0& 0 & 1 & \beta_1 & \beta_2 & \cdots & \beta_{k-1} & \beta_k\\ 
\beta_0& 0 & 1 & \beta_1 & \beta_2 & \cdots & \beta_{k-1} & \beta_k\\ 
0 & 1		& 0 & 0		  & 0   &\cdots &0 &0\\
0& 0		& 0 & 1		  & 0   &\cdots &0 &0\\
0&0		& 0 & 0		  & 1   &\cdots &0 &0\\
\vdots &\vdots & \vdots & \vdots & \vdots & \ddots & \vdots & \vdots\\
0&0	& 0 & 0		  & 0   &\cdots &1 &0
\end{BMAT}
\end{bmatrix}\begin{bmatrix}
\begin{BMAT}[4.5pt]{c}{cc:cccccc}
hf(Y_1)\\
hf(Y_2)\\
y_{n-1}\\
hf(y_{n-1})\\
hf(y_{n-2})\\
hf(y_{n-3})\\
\vdots\\
hf(y_{n-k})
\end{BMAT}
\end{bmatrix}
\end{equation}

\subsubsection{向后微分公式}
向后微分公式(BDF)是第一个被提出来求解刚性问题的数值算法。为了克服Adams算法在求解刚性问题的较小稳定域问题,Curtiss和Hirschfelder在1952年提出了向后微分公式\cite{Curtiss1952}。后来经过Gear的推广\cite{Gear1971a},向后微分公式得到广泛的应用。

向后微分公式可以由格式(\ref{eq:ch2lms})中,令
\begin{equation}
\beta_j=0,\qquad j>0
\end{equation}
得到。亦即逼近解仅仅依赖于当前步的导数值$f(y_n)$。其算法格式为
\begin{equation}
y_n=\sum_{j=1}^{k}\alpha_jy_{n-j}+h\beta_0f(y_n)\label{eq:ch2BDF}
\end{equation}
特别地,向后微分公式的精度可以达到7阶。7阶更高的向后微分公式是不稳定的\cite{Gear1971a}。其中,仅仅只有$k=1$和$k=2$是具有A稳定的。对于其他$k$值事实上不在适合求解刚性问题。对于$k=1\to 6$,其具体的算法格式如下
\begin{align}
k=1:y_n&=y_{n-1}+hf(y_n)\\
k=2:y_n&=\frac{4}{3}y_{n-1}-\frac13y_{n-2}+\frac23hf(y_n)\\
k=3:y_n&=\frac{18}{11}y_{n-1}-\frac{9}{11}y_{n-2}+\frac{2}{11}y_{n-3}+\frac{6}{11}hf(y_{n})\\
k=4:y_n&=\frac{48}{25}y_{n-1}-\frac{36}{25}y_{n-2}+\frac{16}{25}y_{n-3}-\frac{3}{25}y_{n-4}+\frac{12}{25}hf(y_n)\\
k=5:y_n&=\frac{300}{137}y_{n-1}-\frac{300}{137}y_{n-2}+\frac{200}{137}y_{n-3}-\frac{75}{137}y_{n-4}+\frac{12}{137}y_{n-5}+\frac{60}{137}hf(y_n)\\
k=6:y_n&=\frac{120}{49}y_{n-1}-\frac{150}{49}y_{n-2}+\frac{400}{147}y_{n-3}-\frac{75}{49}y_{n-4}+\frac{24}{49}y_{n-5}-\frac{10}{147}y_{n-6}+\frac{20}{49}hf(y_n)
\end{align}

在广义线性法的表示形式下有
\begin{equation}
\begin{bmatrix}
\begin{BMAT}[4.5pt]{c}{c:cccccc}
Y_1\\
y_n\\
hf(y_n)\\
hf(y_{n-1})\\
hf(y_{n-2})\\
\vdots\\
hf(y_{n-k+1})
\end{BMAT}
\end{bmatrix}=\begin{bmatrix}
\begin{BMAT}[5pt]{c:cccccc}{c:cccccc}
\beta_0 & \alpha_1 & \alpha_2 & \alpha_3 & \cdots & \alpha_{k-1} & \alpha_k\\
\beta_0 & \alpha_1 & \alpha_2 & \alpha_3 & \cdots & \alpha_{k-1} & \alpha_k\\
0		& 1 & 0		  & 0   &\cdots &0 &0\\
0		& 0 & 1		  & 0   &\cdots &0 &0\\
0		& 0 & 0		  & 1   &\cdots &0 &0\\
\vdots & \vdots & \vdots & \vdots & \ddots & \vdots & \vdots\\
0	& 0 & 0		  & 0   &\cdots &1 &0
\end{BMAT}
\end{bmatrix}\begin{bmatrix}
\begin{BMAT}[4.5pt]{c}{c:cccccc}
hf(Y_1)\\
y_{n-1}\\
hf(y_{n-1})\\
hf(y_{n-2})\\
hf(y_{n-3})\\
\vdots\\
hf(y_{n-k})
\end{BMAT}
\end{bmatrix}
\end{equation}
其中,$Y_1=y_n$。同时,$k=1$对应向后微分的隐式Euler算法。其广义线性法的表示形式为
\begin{equation}
\begin{bmatrix}
\begin{BMAT}[5pt]{c:c}{c:c}
1 & 1\\ 1 & 1
\end{BMAT}
\end{bmatrix}
\end{equation}
\subsection{Runge-Kutta法}
对于求解微分方程(\ref{eq:ch2FirstAuto})的一般形式的Runge-Kutta方法可描述如下
\begin{subequations}
\begin{align}
Y_i^{[n]}&=y_{n-1}+h\sum_{j=1}^{s}a_{ij}f(Y_j^{[n]}),\quad i=1,2,\cdots,s\\
y_n&=y_{n-1}+h\sum_{j=1}^{s}b_jf(Y_j^{[n]})
\end{align}
\end{subequations}
一般情况下,Runge-Kutta法可通过Butcher表格表示,即
\begin{equation}
\begin{BMAT}[5pt]{c|c}{c|c}
c & A\\  & b^T
\end{BMAT}=\begin{BMAT}[3pt]{c|ccc}{ccc|c}
c_1 & a_{11} & \cdots & a_{1s}\\
\vdots & \vdots & \ddots &\vdots \\
c_s & a_{s1}&\cdots & a_{ss}\\
 & b_1 & \cdots & b_s
\end{BMAT}
\end{equation}
其中,向量$c$表示在单步内极值的位置;而矩阵$A$表示极值对其他极值导数的依赖性;向量$b$表明了积分权重值,暗示了最后的输出量如何依赖于先前计算的极导数值。当然,也可以通过广义线性法的形式表出,即
\begin{equation}
\begin{bmatrix}
\begin{BMAT}[5pt]{c:c}{c:c}
A & e\\ b^T & 1
\end{BMAT}
\end{bmatrix}=\begin{bmatrix}
\begin{BMAT}[3pt]{ccc:c}{ccc:c}
a_{11} & \cdots & a_{1s} & 1\\
\vdots & \ddots & \vdots & \vdots \\
a_{s1} & \cdots & a_{ss} & 1\\
b_1 & \cdots & b_s & 1
\end{BMAT}
\end{bmatrix}\label{eq:ch2GLM2RK}
\end{equation}

有意思的是,前述提及的Euler法也可以看作一阶的Runge-Kutta法,其Butcher表格为
\begin{equation}
\begin{BMAT}[5pt]{c|c}{c|c}
0 & \\
 & 1
\end{BMAT}
\end{equation}
而一般性的二阶精度带任意参数$\theta$的Runge-Kutta方法可用Butcher表格描述为
\begin{equation}
\begin{BMAT}[5pt]{c|cc}{cc|c}
0 & & \\ \theta & \theta & \\
 & 1-\frac{1}{2\theta} & \frac{1}{2\theta}
\end{BMAT}
\end{equation}
当参数$\theta$取值分别为$\frac{1}{2}$和$1$,分别对应中点公式和Trapeoidal规则。在Runge-Kutta的框架下,这二者分别有RK22和RK21表示。其对应的Butcher表格为
\begin{equation}
\text{RK21:}\quad \begin{BMAT}[5pt]{c|cc}{cc|c}
0 & & \\ 1 & 1 & \\
 & \frac{1}{2} & \frac{1}{2}
\end{BMAT}
\end{equation}
\begin{equation}
\text{RK22:}\quad \begin{BMAT}[5pt]{c|cc}{cc|c}
0 & & \\ \frac{1}{2} & \frac{1}{2} & \\
 & 0 & 1
\end{BMAT}
\end{equation}
而我们最常使用的四阶Runge-Kutta方法,也就是MATLAB软件中ode求解器中的“ode45”,可表示为
\begin{equation}
\text{RK41:}\quad \begin{BMAT}[4pt]{c|cccc}{cccc|c}
0 & & & &\\
\frac12 & \frac12 & & &\\
\frac12 & 0 & \frac12 & & \\
1 & 0 & 0 & 1 & \\
 & \frac{1}{6} & \frac13 & \frac13 & \frac16
\end{BMAT}
\end{equation}

由等式(\ref{eq:ch2GLM2RK})不难将这些Runge-Kutta方法使用广义线性法的形式表出,即
\begin{equation}\text{RK21:}\quad 
\begin{bmatrix}
\begin{BMAT}[4pt]{cc:c}{cc:c}
0 & 0& 1\\ 1 & 0 &1 \\
\frac{1}{2} & \frac{1}{2} &1
\end{BMAT}
\end{bmatrix}
\end{equation}
\begin{equation}\text{RK41:}\quad 
\begin{bmatrix}
\begin{BMAT}[3.5pt]{cccc:c}{cccc:c}
0 & 0 & 0 & 0 & 1\\
\frac12 & 0 & 0 & 0 & 1\\
0 & \frac12 & 0 & 0 & 1\\
0 & 0 & 1 & 0 & 1\\
\frac16 & \frac13 & \frac13 & \frac16 & 1\\
\end{BMAT}
\end{bmatrix}
\end{equation}
关于Runge-Kutta方法更加详细的分析和应用可以参考文献或专著\inlinecite{Butcher2008,Lapidus1971,Hairer1993,ErnstHairer1996,李寿佛2010,Lambert1973,Butcher1987}。
\subsection{One-leg方法}
为了简化线性多步法(\ref{eq:ch2lms})在求解刚性问题时的误差分析,Dahlquist引入了如下格式的One-leg方法\cite{Dahlquist1976,Dahlquist1983}。
\begin{equation}
y_n=\sum_{j=1}^{k}\alpha_jy_{n-j}+h\beta f\left(\frac{1}{\beta}\sum_{j=0}^{k}\beta_jy_{n-j}\right)\label{eq:ch2Oneleg}
\end{equation}
其中,参数$\beta$满足$\beta=\sum_{j=0}^{k}\beta_j$。因此,从公式(\ref{eq:ch2Oneleg})可以知道,每步内只需要计算一次$f$的函数值。同时,也需要提及的时,对于对应的线性多步法(\ref{eq:ch2lms}),One-leg方法可能具有更强的非线性稳定性特点,如G稳定性以及对于非一致性网格上更好的鲁棒性\cite{Hundsdorfer1991}。

令
\begin{equation}
Y^{[n]}=\frac{1}{\beta}\sum_{j=1}^{k}\beta_jy_{n-j}
\end{equation}
将等式(\ref{eq:ch2Oneleg})带入到上式就有
\begin{equation}
\begin{aligned}
Y^{[n]}&=\frac{1}{\beta}\left(\beta_0y_n+\sum_{j=1}^{k}\beta_jy_{n-j} \right)\\
&=\frac{1}{\beta}\left[\beta_0\left(\sum_{j=1}^{k}\alpha_jy_{n-j}+h\beta f(Y^{[n]})\right)+\sum_{j=1}^{k}\beta_jy_{n-j}\right]\\
&=\frac{1}{\beta}\sum_{j=1}^{k}(\beta_0\alpha_j+\beta_j)y_{n-j}+h\beta_0f(Y^{[n]})
\end{aligned}
\end{equation}
当建立如下的递推向量
\begin{equation}
y^{[n]}=[y_n\quad y_{n-1}\quad \cdots \quad y_{n-k+1}]^{\text{T}}
\end{equation}
时,One-leg方法(\ref{eq:ch2Oneleg})就可以用广义线性法表出如下$r=k,s=1$
\begin{equation}
\begin{bmatrix}
\begin{BMAT}[5pt]{c:c}{c:c}
A & U\\
B & V
\end{BMAT}
\end{bmatrix}=\begin{bmatrix}
\begin{BMAT}[5pt]{c:ccccc}{c:ccccc}
\beta_0 & \frac{\beta_0\alpha_1+\beta_1}{\beta} & \frac{\beta_0\alpha_2+\beta_2}{\beta} & \cdots & \frac{\beta_0\alpha_{k-1}+\beta_{k-1}}{\beta} & \frac{\beta_0\alpha_k+\beta_k}{\beta}\\
\beta & \alpha_1 & \alpha_2 & \cdots & \alpha_{k-1} & \alpha_k\\
0 & 1 & 0 & \cdots & 0 & 0 \\
0 & 0 & 1 & \cdots & 0 & 0 \\
\vdots &\vdots & \vdots & \ddots & \vdots & \vdots \\
0 & 0 & 0 & \cdots & 1 & 0 \\
\end{BMAT}
\end{bmatrix}
\end{equation}
关于One-leg方法的另外一种书写方式可以参见文献\inlinecite{Butcher2006c}。

\subsection{扩展的向后微分格式}
Cash提出了扩展到向后微分公式(Extended backward differentiation formulas:EBDFs)\cite{Cash1980}求解刚性微分方法。该格式涉及到$t_{n+k+1}$时刻$f$的计算,其算法可描述为
\begin{equation}
\sum_{j=0}^{k}\alpha_jy_{n+j}=h\beta_kf(y_{n+k})+h\beta_{k+1}f(y_{n+k+1})\label{eq:ch2EBDFs}
\end{equation}
其中,参数$\alpha_j,j=0,1,\cdots,k,\beta_k,\beta_{k+1}$可通过算法格式的$k+1$阶精度条件确定,而不失一般性的可以假定
\begin{equation}
\alpha_k=1
\end{equation}
对于$k=1,2,3$,上述EBDFs算法格式可实现A稳定和L稳定;而对于$k=4,5,6,7,8$,仅仅是$A(\alpha)$稳定的。其稳定域大小可见文\inlinecite{Cash1980,ErnstHairer1996}。

假设变量逼近值$y_n,y_{n+1},\cdots,y_{n+k-1}$已知,则基于EBDF格式的计算步骤可描述为
\begin{itemize}
\item[i.] 基于经典的向后微分格式计算$\overline{y}_{n+k}$
\begin{equation}
\overline{y}_{n+k}+\sum_{j=0}^{k-1}\hat{\alpha}_jy_{n+j}=h\hat{\beta}_kf(\overline{y}_{n+k})
\end{equation}
\item[ii.] 继续用上式中的向后微分公式计算$\overline{y}_{n+k+1}$的值,即
\begin{equation}
\overline{y}_{n+k+1}+\hat{\alpha}_{k-1}\overline{y}_{n+k}+\sum_{j=0}^{k-2}\hat{\alpha}_jy_{n+j+1}=h\hat{\beta}_kf(\overline{y}_{n+k+1})
\end{equation}
\item[iii.] 丢弃$\overline{y}_{n+k}$,在EBDF格式(\ref{eq:ch2EBDFs})中引入$\overline{y}_{n+k+1}$求解$y_{n+k}$可得
\begin{equation}
y_{n+k}+\sum_{j=0}^{k-1}\alpha_jy_{n+j}=h\beta_kf(y_{n+k})+h\beta_{k+1}f(\overline{y}_{n+k+1})
\end{equation}
\end{itemize}
正如文\inlinecite{Cash1983}所指出的,上述EBDF格式的缺点是在步骤(i,ii)中需要求解带有相同雅可比矩阵($I-h\hat{\beta}_kJ,J=\partial f/\partial y$)的非线性方程组,但在第iii步骤中,却又有一个不同的雅可比矩阵($I-h\beta_kJ$)进行非线性求解,这就导致了额外的计算量。为了弥补这个缺点,Cash又提出了修改版的EBDF格式(MEBDF)\cite{Cash1983},即
\begin{equation}
\sum_{j=0}^{k}\alpha_jy_{n+j}=h\hat{\beta}_kf(y_{n+k})+h(\beta_k-\hat{\beta}_k)f(\overline{y}_{n+k})+h\beta_{k+1}f(\overline{y}_{n+k+1})
\end{equation}

数值格式(MEBDF)对于$k=1,2,3$也是A稳定和L稳定的;而对于$k=4,5,6,7,8$都是带有较对应的EBDF格式更大$\alpha$角度的$A(\alpha)$稳定算法。对于BDF、EBDF和MEBDF算法所对应的$\alpha$角度列于表\ref{Tab:ch2AalphaBDF}。
\begin{table}[htbp]
\bicaption[Tab:ch2AalphaBDF]{}{BDF、EBDF和MEBDF算法所对应的$A(\alpha)$稳定的$\alpha$角度值}{Table$\!$}{Angles $\alpha$ of $A(\alpha)$-stability for BDF, EBDF and MEBDF schemes}
\vspace{0.5em}\centering\wuhao
\begin{tabular}{ccccccccc}
\toprule[1.5pt]
k & 1 & 2 & 3 & 4 & 5 & 6 & 7 & 8\\
\midrule[1pt]
BDF & $90^\circ$ & $90^\circ$ & $88^\circ$ & $73^\circ$ & $51^\circ$ & $18^\circ$ & $*$ & $*$\\
EBDF & $90^\circ$ & $90^\circ$ & $90^\circ$ & $87.61^\circ$ & $80.21^\circ$ & $67.73^\circ$ & $48.82^\circ$ & $19.98^\circ$\\
MEBDF & $90^\circ$ & $90^\circ$ & $90^\circ$ & $88.36^\circ$ & $83.07^\circ$ & $74.48^\circ$ & $61.98^\circ$ & $42.87^\circ$\\
\bottomrule[1.5pt]
\end{tabular}
\end{table}
注意,表格中带有“$*$”表示不是$A(\alpha)$稳定的。

将步骤i中的$\overline{y}_{n+k}$带入到步骤ii中的方程中,得
\begin{equation}
\begin{aligned}
\overline{y}_{n+k+1}=&\hat{\alpha}_{k-1}\hat{\alpha}_0y_n+\sum_{j=1}^{k-1}(\hat{\alpha}_{k-1}\hat{\alpha}_j-\hat{\alpha}_{j-1})y_{n+j}\\
&-h\hat{\alpha}_{k-1}\hat{\beta}_{k}f(\overline{y}_{n+k})+h\hat{\beta}_{k}f(\overline{y}_{n+k+1})
\end{aligned}
\end{equation}
于是,基于MEBDF格式,其广义线性法的表示形式中的四个矩阵可写为
\begin{equation}
Y^{[n]}=\begin{bmatrix}
\overline{y}_{n+k}\\
\overline{y}_{n+k+1}\\
{y}_{n+k}
\end{bmatrix},\quad F(Y^{[n]})=\begin{bmatrix}
f(\overline{y}_{n+k})\\
f(\overline{y}_{n+k+1})\\
f({y}_{n+k})\\
\end{bmatrix},\quad y^{[n]}=\begin{bmatrix}
y_{n+k}\\
y_{n+k-1}\\
\vdots\\
y_{n+1}
\end{bmatrix}
\end{equation}
其系数矩阵$A,B,U$和$V$分别为
\begin{subequations}
\begin{align}
& \qquad \qquad \qquad\quad A=\begin{bmatrix}
\hat{\beta}_k & 0 & 0\\
-\hat{\alpha}_{k-1}\hat{\beta}_{k} & \hat{\beta}_{k} & 0\\
\beta_k-\hat{\beta}_k & \beta_{k+1} & \hat{\beta}_k
\end{bmatrix}\\
U &= \begin{bmatrix}
-\hat{\alpha}_{k-1} &-\hat{\alpha}_{k-2} & \cdots & -\hat{\alpha}_1 & -\hat{\alpha}_0\\
\hat{\alpha}_{k-1}\hat{\alpha}_{k-1}-\hat{\alpha}_{k-2} & \hat{\alpha}_{k-1}\hat{\alpha}_{k-2}-\hat{\alpha}_{k-3} & \cdots & \hat{\alpha}_{k-1}\hat{\alpha}_1-\hat{\alpha}_0 & \hat{\alpha}_{k-1}\hat{\alpha}_0\\
-\alpha_{k-1} & -\alpha_{k-2} & \cdots & -\alpha_1 & -\alpha_0
\end{bmatrix}\\
B&=\begin{bmatrix}
\beta_k-\hat{\beta}_k & \beta_{k+1} & \hat{\beta}_k\\
0 & 0 & 0\\
\vdots & \vdots & \vdots \\
0 & 0 & 0\\
0 & 0 & 0
\end{bmatrix},\qquad V=\begin{bmatrix}
-\alpha_{k-1} & -\alpha_{k-2} & \cdots & -\alpha_1 & -\alpha_0\\
1 & 0 & \cdots & 0 & 0\\
\vdots & \vdots & \ddots & \vdots & \vdots \\
0 & 0 & \cdots & 0 & 0\\
0 & 0 & \cdots & 1 & 0\\
\end{bmatrix}
\end{align}
\end{subequations}
\subsection{两步Runge-Kutta方法}
考虑文\inlinecite{Jackiewicz1995a}中提出的两步Runge-Kutta方法(Two-step Runge-Kutta: TSRK)。该类算法依赖于两个连续步的极数值,其算法可以描述为
\begin{subequations}
\begin{align}
Y_i^{[n]}&=(1-\mu_i)y_{n-1}+\mu_iy_{n-2}+h\sum_{j=1}^{s}\left(a_{ij}f(Y_j^{[n]})+b_{ij}f(Y_j^{[n-1]})\right)\\
y_n&=(1-\vartheta)y_{n-1}+\vartheta y_{n-2}+h\sum_{j=1}^{s}\left(v_jf(Y_j^{[n]})+w_jf(Y_j^{[n-1]}) \right)
\end{align}\label{eq:ch2TSRK}
\end{subequations}
其中,$i=1,2,\cdots,s$,且$Y_i^{[n]}$是$y(t_{n-1}+c_ih)$的逼近值。TSRK也可以由下列的Butcher表格给出
\begin{equation}
\begin{BMAT}[5pt]{c|c|c}{c|c}
\bm{u} & \bm{A} & \bm{B}\\
\vartheta & \bm{v}^T & \bm{w}^T
\end{BMAT}
\end{equation}
在广义线性法的表示形式下,则有$r=s+2$
\begin{equation}
\begin{bmatrix}
\begin{BMAT}[4.5pt]{c}{c:ccc}
Y^{[n]}\\ y_n\\ y_{n-1}\\ hF(Y^{[n]})
\end{BMAT}
\end{bmatrix}=\begin{bmatrix}
\begin{BMAT}[5pt]{c:ccc}{c:ccc}
\bm{A} & \bm{e-u} & \bm{u} & \bm{B}\\
\bm{v}^T & 1-\vartheta & \vartheta & \bm{w}^T\\
0 & 1 & 0 & 0\\
\bm{I} & \bm{0} & \bm{0} & \bm{0}
\end{BMAT}
\end{bmatrix}\begin{bmatrix}
\begin{BMAT}[4.2pt]{c}{c:ccc}
hF(Y^{[n]})\\ y_{n-1}\\ y_{n-2}\\ hF(Y^{[n-1]})
\end{BMAT}
\end{bmatrix}
\end{equation}
其中,$\bm{I}$是维度为$s$的单位矩阵,而$\bm{0}$是带有合适维度的全零矩阵,$\bm{A}=[a_{ij}]_{s\times s},\bm{B}=[b_{ij}]_{s\times s},\bm{v}=[v_j]_{s\times 1},\bm{w}=[w_j]_{s\times 1}$,而$\bm{e}$为$s$维元素全为1的列向量。

\subsection{多步Runge-Kutta方法}
Burrage和Sharp等人研究如下形式的多步Runge-Kutta方法(Multistep Runge-Kutta: MRK)\cite{Burrage1994,Burrage1978a,Burrage1988}。
\begin{subequations}
\begin{align}
Y_i^{[n]}&=h\sum_{j=1}^{s}a_{ij}f(Y_j^{[n]})+\sum_{j=1}^{k}u_{ij}y_{n+1-j}\\
y_{n+1}&=h\sum_{j=1}^{s}b_jf(Y_j^{[n]})+\sum_{j=1}^{k}v_{j}y_{n+1-j}
\end{align}\label{eq:ch2MRK}
\end{subequations}
其中,$i=1,2,\cdots,s$。而$Y_i^{[n]}$是$y(t_n+c_ih)$的逼近值。对于$k=1$,上述的MRK方法就退化为普通的Runge-Kutta算法;而对于$k=2$,MRK方法并不能退化为前述提及的TSRK方法(\ref{eq:ch2TSRK})。这是因为MRK方法仅仅依{}赖于当前步内的级数值$Y_i^{[n]}$,而TSRK方法却依赖于连续两个时间步长内的级数值$Y_i^{[n]},Y_i^{[n-1]}$。

通过令
\begin{equation}
y^{[n]}=[y_{n+1}\quad y_n\quad \cdots\quad y_{n-k+2}]^T
\end{equation}
MRK方法(\ref{eq:ch2MRK})可以写成广义线性法的形式如下
\begin{equation}
\begin{bmatrix}
\begin{BMAT}[5pt]{c:c}{c:c}
A & U\\
B & V
\end{BMAT}
\end{bmatrix}=\begin{bmatrix}
\begin{BMAT}[5pt]{cccc:ccccc}{cccc:ccccc}
a_{11} & a_{12} & \cdots & a_{1s} & u_{11} & u_{12} & \cdots & u_{1,k-1} & u_{1k}\\
a_{21} & a_{22} & \cdots & a_{2s} & u_{21} & u_{22} & \cdots & u_{2,k-1} & u_{2k}\\
\vdots & \vdots & \ddots & \vdots & \vdots & \vdots & \ddots & \vdots & \vdots \\
a_{s1} & a_{s2} & \cdots & a_{ss} & u_{s1} & u_{s2} & \cdots & u_{s,k-1} & u_{sk}\\
b_1 & b_2 & \cdots & b_s & v_1 & v_2 & \cdots & v_{k-1} & v_k\\
0 & 0 & \cdots & 0 & 1 & 0 & \cdots & 0 & 0\\
\vdots & \vdots & \ddots & \vdots & \vdots & \vdots & \ddots & \vdots & \vdots \\
0 & 0 & \cdots & 0 & 0 & 0 & \cdots & 0 & 0\\
0 & 0 & \cdots & 0 & 0 & 0 & \cdots & 1 & 0
\end{BMAT}
\end{bmatrix}
\end{equation}
\subsection{Peer方法}
Weiner等人提出了Peer方法\cite{Schmitt2004},该方法使得所有的级数值都有相同的特点,同时没有额外的求解向量使用。对于两步Peer方法,在一致网格划分下有如下形式
\begin{equation}
Y_i^{[n]}=\sum_{j=1}^{s}b_{ij}Y_j^{[n]}+h\sum_{j=1}^{s}a_{ij}f(Y_j^{[n-1]})+h\sum_{j=1}^{s}r_{ij}f(Y_j^{{n}})
\end{equation}
其中,$i=1,2,\cdots,s$,在向量形式下有
\begin{equation}
Y^{[n]}=(\bm{B}\otimes \bm{I})Y^{[n-1]}+h(\bm{A}\otimes \bm{I})F(Y^{[n-1]})+h(\bm{R}\otimes \bm{I})F(Y^{[n]})
\end{equation}
而$Y^{[n]}$是$y(t_n+c_ih),i=1,2,\cdots,s$的逼近值。上述格式在广义线性法的形式下表出为
\begin{equation}
\begin{bmatrix}
\begin{BMAT}[4.5pt]{c}{c:cc}
Y^{[n]}\\ Y^{[n]}\\ hF(Y^{[n]})
\end{BMAT}
\end{bmatrix}=\begin{bmatrix}
\begin{BMAT}[5pt]{c:cc}{c:cc}
\bm{R} & \bm{B} & \bm{A}\\
\bm{R} & \bm{B} & \bm{A}\\
\bm{I} & \bm{0} & \bm{0}
\end{BMAT}
\end{bmatrix}\begin{bmatrix}
\begin{BMAT}[4.2pt]{c}{c:cc}
hF(Y^{[n]})\\ Y^{[n-1]}\\ hF(Y^{[n-1]})
\end{BMAT}
\end{bmatrix}
\end{equation}

%================================================================================================
\section{相容性}
这一节,将建立广义线性法(\ref{eq:ch2GLM})的相容性条件。当然,从广义线性法的相容条件可以退化为线性多步法和Runge-Kutta法的相容性条件。假设下列两个向量存在
\begin{align}
\bm{q}_0=[q_{1,0}\quad {q}_{2,0}\quad \cdots \quad{q}_{r,0}]^T\\
\bm{q}_1=[q_{1,1}\quad {q}_{2,1}\quad \cdots \quad{q}_{r,1}]^T
\end{align}
使得输入向量$y^{[n-1]}$满足
\begin{equation}
y_i^{[n-1]}=q_{i,0}y(t_{n-1})+q_{i,1}hy'(t_{n-1})+\mathcal{O}(h^2),\quad i=1,2,\cdots,r
\end{equation}
同时,要求级数值$Y^{[n]}$和输出向量$y^{[n]}$的分量满足
\begin{align}
Y^{[n]}_i&=y(t_{n-1}+c_ih)+\mathcal{O}(h^2),\quad i=1,2,\cdots,s\\
y_i^{[n]}&=q_{i,0}y(t_{n})+q_{i,1}hy'(t_{n})+\mathcal{O}(h^2),\quad i=1,2,\cdots,r
\end{align}

正如文\inlinecite{Butcher2008,Burrage1995,Jackiewicz2009}所说,将这些关系带入广义线性法(\ref{eq:ch2GLM})可得
\begin{equation}
\begin{aligned}
y(t_{n-1})+&hc_iy'(t_{n-1})=h\sum_{j=1}^{s}a_{ij}y'(t_{n-1})\\
&+\sum_{j=1}^{r}u_{ij}\left(q_{j,0}y(t_{n-1})+hq_{j,1}y'(t_{n-1})\right)+\mathcal{O}(h^2),i=1,2,\cdots,s
\end{aligned}
\end{equation}\vspace{-0.5cm}
\begin{equation}
\begin{aligned}
q_{i,0}y(t_{n})+&q_{i,1}hy'(t_{n})=h\sum_{j=1}^{s}b_{ij}y'(t_{n})\\
&+\sum_{j=1}^{r}v_{ij}\left(q_{j,0}y(t_{n-1})+hq_{j,1}y'(t_{n-1})\right)+\mathcal{O}(h^2),i=1,2,\cdots,r
\end{aligned}
\end{equation}
在上述两式总比较$\mathcal{O}(1)$和$\mathcal{O}(h)$项的系数可得
\begin{equation}
\sum_{j=1}^{r}u_{ij}q_{j,0}=1,\ i=1,2,\cdots,s\qquad \sum_{j=1}^{r}v_{ij}q_{j,0}=q_{i,0},\ i = 1,2,\cdots,r
\end{equation}
和
\begin{subequations}
\begin{align}
\sum_{j=1}^{s}a_{ij}+\sum_{j=1}^{r}u_{ij}q_{j,1}&=c_i,\ i=1,2,\cdots,s\\
\sum_{j=1}^{s}b_{ij}+\sum_{j=1}^{r}v_{ij}q_{j,1}&=q_{i,0}+q_{i,1},\ i=1,2,\cdots,r
\end{align}
\end{subequations}
于是,就有如下定义
\begin{definition}[预相容]
广义线性法($\bm{c,A,U,B,V}$)是预相容的\cite{Jackiewicz2009,Burrage1995,Butcher2008},如果存在向量$\bm{q}_0$使得
\begin{equation}
\bm{Uq}_0=\bm{e},\quad \bm{Vq}_0=\bm{q}_0\label{eq:ch2PreConsistent}
\end{equation}
其中,$\bm{e}=[1,\cdots,1]^T\in\mathbb{R}^s$。而向量$\bm{q}_0$称为预相容向量。
\end{definition}










考虑一个非常简单的一阶微分方程
\begin{equation}
y'=0,\quad y(0)=y_0\label{eq:ch2FirstTestPro}
\end{equation}
显然,该问题的有精确的显式解$y(t)=y_0$。亦即对任何的数值算法求解该问题(\ref{eq:ch2FirstTestPro})都应该有获得有界的数值逼近解$y_n$。否则,若设计一个数值算法连求解(\ref{eq:ch2FirstTestPro})都存在发散问题,可能这样的设计也是失败的。于是,针对广义线性法(\ref{eq:ch2GLM})求解该问题有
\begin{equation}
y^{[n+1]}=Vy^{[n]}
\end{equation}
若想使广义线性法(\ref{eq:ch2GLM})求解该问题获得有界解,则要求矩阵$V$满足一定的要求。








\section{稳定性}

\section{收敛性}

\section{精度分析}

\subsection{数值耗散}

\subsection{数值弥散}

\section{超调行为}

\section{虚假根分析}

\section{直接积分法的放大矩阵}

\begin{definition}
直接积分算法的一致性\cite{Hoff1988}。如果两个直接积分算法的放大矩阵$A$的对应元素分别相等\footnote{这也是矩阵相等的定义。},则称这两个积分算法是一致的。
\end{definition}

\begin{definition}
直接积分算法的相似性/谱等价\cite{Hoff1988}。如果两个直接积分算法的放大矩阵$A$的不变量分别对应相等,则称这两个积分算法是相似的/谱等价的。
\end{definition}

需要说明的是,一般情况下,两个谱等价的算法并不一定能导出一致的数值算法特性;而两个算法的一致性可以保证。这样的例子可以参见文献\inlinecite{Hoff198887}。

\section{本章小结}



